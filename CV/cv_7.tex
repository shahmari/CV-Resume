%----------------------------------------------------------------------------------------
%	PACKAGES AND OTHER DOCUMENT CONFIGURATIONS
%----------------------------------------------------------------------------------------

\documentclass[11pt,a4paper,sans]{moderncv} % Font sizes: 10, 11, or 12; paper sizes: a4paper, letterpaper, a5paper, legalpaper, executivepaper or landscape; font families: sans or roman

\moderncvstyle{casual} % CV theme - options include: 'casual' (default), 'classic', 'oldstyle' and 'banking'
\moderncvcolor{blue} % CV color - options include: 'blue' (default), 'orange', 'green', 'red', 'purple', 'grey' and 'black'

\usepackage{lipsum} % Used for inserting dummy 'Lorem ipsum' text into the template

\usepackage[scale=0.83]{geometry} % Reduce document margins


%\setlength{\hintscolumnwidth}{3cm} % Uncomment to change the width of the dates column
%\setlength{\makecvtitlenamewidth}{10cm} % For the 'classic' style, uncomment to adjust the width of the space allocated to your name

%----------------------------------------------------------------------------------------
%	NAME AND CONTACT INFORMATION SECTION
%----------------------------------------------------------------------------------------

\firstname{Yaghoub} % Your first name
\familyname{Shahmari} % Your last name

% All information in this block is optional, comment out any lines you don't need
\title{Academic Resume}
%\address{24//168 Lo Duc St.}{Hamirpur, Himachal Pradesh}
%\mobile{(+91) 9882541575}
%\phone{(000) 111 1112}
%\fax{(000) 111 1113}
%\email{clearnote01@gmail.com}
%\homepage{staff.org.edu/~jsmith}{staff.org.edu/$\sim$jsmith} % The first argument is the url for the clickable link, the second argument is the url displayed in the template - this allows special characters to be displayed such as the tilde in this example
%\extrainfo{additional information}
%\photo[70pt][0.4pt]{pictures/picture} % The first bracket is the picture height, the second is the thickness of the frame around the picture (0pt for no frame)
%\quote{"Faith sees a beautiful blossom in a bulb" - William Arthur Ward}

%----------------------------------------------------------------------------------------

\begin{document}

\makecvtitle % Print the CV title

%-----------------
%   Contact Information
%-----------------

\section{Contact Information}

\cvitem{\textbf{}Email:}{shahmari.acer@gmail.com \textit{} }
\cvitem{\textbf{}Phone:}{+98 922 643 4531 \textit{} }
\cvitem{\textbf{}Website:}{
				https://physics.sharif.edu/\textasciitilde shahmari.acer
                }
\cvitem{}{https://www.linkedin.com/in/yaghoub-shahmari/ \textit{} }


%\cventry{2008--Present}{Bachelor of Computer Science}{Hanoi University of Science and Technology}{School of Information and Communication Technology, Hanoi.}{\textit{GPA -- 8.0}}{}  % Arguments not required can be left empty
%\cvitem{}{Class ICT56 -- ICT Program: program for Gifted Students; the curriculum is conducted in \textbf{English}.}

\renewcommand{\listitemsymbol}{~~~~} % Changes the symbol used for lists






%----------------------------------------------------------------------------------------
%	EDUCATION SECTION
%----------------------------------------------------------------------------------------

\section{Education}

\cvitem{2019--Present}{\textbf{Bachelor of Physics and minor of Computer Science:} \href{https://en.sharif.edu/}{Sharif University of Technology}, Tehran }
%\cventry{2008--Present}{Bachelor of Computer Science}{Hanoi University of Science and Technology}{School of Information and Communication Technology, Hanoi.}{\textit{GPA -- 8.0}}{}  % Arguments not required can be left empty
%\cvitem{}{Class ICT56 -- ICT Program: program for Gifted Students; the curriculum is conducted in \textbf{English}.}
\renewcommand{\listitemsymbol}{~~~~} % Changes the symbol used for lists
\cvlistdoubleitem{CGPA: 16.82/20}{Highest SGPA: 18.31/20}

\cvitem{2012-2019}{\textbf{MIDDLE-SCHOOL AND HIGH-SCHOOL:} \href{https://sampad.gov.ir/}{NODET} Schools (National Organization for Development of Exceptional Talents), Tehran}

%--------------------------------------
%  Interests
%--------------------------------------

\section{Areas of Interest}

\cvitem{\textbf{}}{Nonlinear Dynamic, Complex Systems, Modeling Infectious Diseases, Neuroscience, }
\cvitem{}{Modeling Quantitive systems, Data Science, Computational science, Econophysics}
%\cvitem{\textbf{}}{, Software Development}
%/cvitem{\textbf{}}{}



%--------------------------------------
%  Projects
%--------------------------------------


\section{Projects}

\cvitem{December 2020}{\textbf{Simulation of Mechanical Random Walker System:} This was my first computational project and simulation of a physical system. After I watched a video of a \href{https://www.youtube.com/shorts/Vo9Esp1yaC8}{Galton Boards} device, I was motivated to simulate it using code. My code simulates the same system and condition in the video. I presented this project as the Analytical Mechanic 1 (Presented by Dr. Rahvar) term project. }
\renewcommand{\listitemsymbol}{} % Changes the symbol used for lists
\cvlistitem{\textbf{\href{https://github.com/shahmari/Random-Walk-particle-system}{Github}}}
\cvitem{May 2021}{\textbf{Simulation of Janus bunch:} I was looking for a topic for the term project of Analytical Mechanic 2 (Presented by Dr. Rahvar) that is related to complex systems fields. I started this after I recieved some advice from Dr. Saman Moghim. }
\renewcommand{\listitemsymbol}{} % Changes the symbol used for lists
\cvlistitem{\textbf{\href{https://github.com/shahmari/Synchronization-in-janus-bunch}{Github}}}

\cvitem{May 2021}{\textbf{Investigating the effects of Prevention and Quarantine on SIR:} It was my term project of the Modeling Statistical Phenomena course under the supervision of Dr. F Ghanbarnejad. The algorithm was simple but the challenge was optimization. The aim of this simulation was to apply the algorithm on a very large network and I couldn't let it run forever! this was reproduction of the \href{https://www.nature.com/articles/srep00010}{original paper}.}
\renewcommand{\listitemsymbol}{} % Changes the symbol used for lists
\cvlistitem{\textbf{\href{https://github.com/shahmari/Prevention-and-Quarantine-on-SIR}{Github}}}

\cvitem{January 2022}{\textbf{Neuroscience term project:} It was my group's term project of the Introduction to Neuroscience (Presented by Dr. Ghazizadeh). In this project, we investigated the data of a monkey during behavioral tasks. The whole experiment contained many trials and throughout the experiment, an electrode stored data of motor parts of the brain.}
\renewcommand{\listitemsymbol}{} % Changes the symbol used for lists
\cvlistitem{\textbf{\href{https://github.com/shahmari/Random-Walk-particle-system}{Github}}}
\cvlistitem{Contributor: Maedeh Karkhaneh Yousefi, Ali Abolhassanzadeh Mahani}

\cvitem{March 2022 - Present}{\textbf{Modeling of commodity prices:} Modeling of commodity prices, relating the field of Complex Systems with Economy,  both analytically and computationally.}
\renewcommand{\listitemsymbol}{} % Changes the symbol used for lists
\cvlistitem{\textbf{\href{https://github.com/shahmari/Modeling-of-Commodity-Prices}{Github}}}
\cvlistitem{Contributor: Maedeh Karkhaneh Yousefi}


\cvitem{}{}


%--------------------------------------
%  SELECTED ASSIGNMENTS
%--------------------------------------


\section{Selected Assignments}

\cvitem{Spring 2021}{\textbf{Some minor efforts in the epidemiology:} They are all of my saved efforts as homework and minor projects of the Modeling Statistical Phenomena course. Due to lack of my attention, Most of my efforts are lost forever. }
\renewcommand{\listitemsymbol}{} % Changes the symbol used for lists
\cvlistitem{\textbf{\href{https://github.com/shahmari/Some-minor-effort-in-the-epidemic}{Github}}}
\cvitem{Fall 2021}{\textbf{Computational Physics 2021:} They are all of my efforts as homework and minor projects of the Computational Physics course (Computer Simulation in Physics). During this course, I learned Julia programming language myself and used it to do homework. }
\renewcommand{\listitemsymbol}{} % Changes the symbol used for lists
\cvlistitem{\textbf{\href{https://github.com/shahmari/ComputationalPhysics-Fall2021}{Github}}}

\cvitem{}{}


%----------------------------------------------------------------------------------------
%	SKILLS SECTION
%----------------------------------------------------------------------------------------
%\vspace{14mm}
\section{Technical Skills}

\subsection{Languages}

\cvitem{} {\textsc{Julia, Python, C, C++, C\#, HTML, CSS,}}
\cvitem{}{\textsc{Octave, Matlab, Mathematica, Maple}}

\subsection{Software}

\cvitem{} {\textsc{Linux}, \textsc{VSCode}, \textsc{Git}, \textsc{\LaTeX},}
\cvitem{} {Unity, Photoshop, Solidworks, Catia, Word, Power Point, Excel}

\cvitem{}{}
\section{Selected Courses}
%---------------------------------------------------------------------
%	SELECTED COURSES
%------------------------------------------------------------------------

\cvitem{Spring 2020}{Introduction to Universe (Dr. B.Mashhoon) - Grade: 18.0/20}
\cvitem{Spring 2021}{Modeling Statistical Phenomena (Dr. F.Ghanbarnejad) - Grade: 16.5/20}
\cvitem{Spring 2021}{Thermodynamic and Statistical Physics 1 (Dr. O.Akhavan) - Grade: 19/20}
\cvitem{Fall 2021}{Thermodynamic and Statistical Physics 2 (Dr. O.Akhavan) - Grade: 19.8/20}
\cvitem{Fall 2021}{Computer Simulation in Physics (Dr. M.R.Ejtehadi) - Grade: 17.5/20}
\cvitem{Fall 2021}{Computer Simulation lab in Physics (Dr. M.R.Ejtehadi) - Grade: 17.5/20}
\cvitem{Fall 2021}{Introduction to Neuroscience (Dr. A.Ghazizadeh) - Grade: 14.7/20}
\cvitem{Spring 2022}{Complex Systems (Dr. S.Ruhani) - Grade: N/A}
\cvitem{Spring 2022}{Data Science \& HPC (Dr. H.R.Arian) - Grade: N/A}
\cvitem{Spring 2022}{Numerical Analysis (Dr. N.Mahdavi) - Grade: N/A}

%---------------------------------------------------------------------
%	EXTRACURRICULAR COURSES
%------------------------------------------------------------------------
% \cvitem{}{}
% \section{EXTRACURRICULAR Courses}

% \cvitem{}{Introduction to LaTeX (Organizer: SPSSA)}
% \cvitem{}{Introduction to Python (Organizer: SPSSA)}
% \cvitem{}{Introductory Python (Organizer: Maktabkhooneh)}
% \cvitem{}{Advanced Python Course (Organizer: Maktabkhooneh)}
% \cvitem{}{Sharif Socio-Physics School (SUT Physics Department)}

% %----------------------------
%   Extracurricular Ativities
%----------------------------

\pagebreak
\cvitem{}{}
\section{Voluntary Activities}
\cvitem{2020-2021}{\textbf{Main Member of the Committee, Public relations manager, and Graphist of Sharif Physics Student Scientific Association, \href{https://spssa.ir/}{Link}}}
\cvitem{}{The Physics Student Scientific Association is one of the scientific, cultural and student associations of Sharif University and is under the supervision of the Faculty of Physics. The Scientific Association of the Faculty of Physics has always been trying to organize valuable and fruitful activities and programs as extracurricular activities for students and those interested in physics.}
\cvitem{2020-2021}{\textbf{Head of News-Reading of Department, \href{https://t.me/khabarkhaani}{Link}}}
\cvitem{}{The news-reading of the Department of Physics is a program under the supervision of Dr. Moghimi that examines the daily news of the world of physics and invites students and professors to present the latest news of the world of physics in 15 minutes.}
\cvitem{2020-2021}{\textbf{Member of the Executive Committee, Graphist, and lecturer of Landa Scientific Circle, \href{https://t.me/lambda_circle}{Link}}}
\cvitem{}{The Landa Circle is one of the scientific circles of the Department of Physics that encourages students to give presentations in physical and non-physical fields.}
\cvitem{Summer 2021}{\textbf{Executive member Sharif Socio-Physics School, \href{http://physics.sharif.edu/~ssp2021/}{Link}}}
\cvitem{}{The Sharif Socio-physics School (SSP2021) was held from 14th September to 16th September. The school provided young graduate students and researchers and even advanced undergraduate students a useful set of information and skills on socio-physics.}
\cvitem{Fall 2020}{\textbf{Organizer and Graphist of the Conference introducing the branches of Physics, \href{https://bit.ly/3O7duaF}{Link}}}
\cvitem{}{Organising series of "Introduction to Fields of Physics" sessions, having the Physics department professors as lecturers.}
\cvitem{Spring 2021}{\textbf{Organizer and Graphist of the Session of Introduction to Complex Systems, \href{https://t.me/anjoman_elmi_phys_sut/2062}{Link}}}
\cvitem{}{A session organized for Introducing the filed of Complex systems to the physics students of Sharif University of Technology  by three International students of highest-ranking Europe universities.}
\cvitem{Winter 2022}{\textbf{Lecturer of the SPSSA Julia Workshop, \href{https://bit.ly/3ruO1ht}{Link}}}
\cvitem{}{A workshop suitable for those who want to get acquainted with the capabilities, strengths and features of Julia programming language and change their programming language to Julia.}


%----------------------------------------------------------------------------------------
%	INTERESTS SECTION
%----------------------------------------------------------------------------------------
% \cvitem{}{}
% \section{Recreational Interests}

% \renewcommand{\listitemsymbol}{-~} % Changes the symbol used for lists

% \cvlistdoubleitem{Japanese Animations \& Comics}{Reading historical accounts}
% \cvlistdoubleitem{Photography}{German Language}

% %----------------------------------------------------------------------------------------
%	COVER LETTER
%----------------------------------------------------------------------------------------

% To remove the cover letter, comment out this entire block

%\clearpage

%\recipient{HR Department}{Corporation\\123 Pleasant Lane\\12345 City, State} % Letter recipient
%\date{\today} % Letter date
%\opening{Dear Sir or Madam,} % Opening greeting
%\closing{Sincerely yours,} % Closing phrase
%\enclosure[Attached]{curriculum vit\ae{}} % List of enclosed documents

%\makelettertitle % Print letter title

%\lipsum[1-3] % Dummy text

%\makeletterclosing % Print letter signature

%----------------------------------------------------------------------------------------

\end{document}
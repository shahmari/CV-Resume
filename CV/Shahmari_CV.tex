%----------------------------------------------------------------------------------------
%	PACKAGES AND OTHER DOCUMENT CONFIGURATIONS
%----------------------------------------------------------------------------------------

\documentclass[11pt,a4paper,sans]{moderncv} % Font sizes: 10, 11, or 12; paper sizes: a4paper, letterpaper, a5paper, legalpaper, executivepaper or landscape; font families: sans or roman

\moderncvstyle{casual} % CV theme - options include: 'casual' (default), 'classic', 'oldstyle' and 'banking'
\moderncvcolor{blue} % CV color - options include: 'blue' (default), 'orange', 'green', 'red', 'purple', 'grey' and 'black'

\usepackage{lipsum} % Used for inserting dummy 'Lorem ipsum' text into the template

\usepackage[scale=0.83]{geometry} % Reduce document margins

%----------------------------------------------------------------------------------------
%	NAME AND CONTACT INFORMATION SECTION
%----------------------------------------------------------------------------------------

\firstname{Yaghoub} % Your first name
\familyname{Shahmari} % Your last name

% All information in this block is optional, comment out any lines you don't need
\title{Curriculum vitae}
\mobile{(+98) 922 6434 531}
%\phone{(000) 111 1112}
%\fax{(000) 111 1113}
\email{shahmari.acer@gmail.com}
\homepage{www.linkedin.com/in/yaghoub-shahmari/}{Linkedin} % The first argument is the url for the clickable link, the second argument is the url displayed in the template - this allows special characters to be displayed such as the tilde in this example
%\extrainfo{additional information}
% \photo[70pt][0.4pt]{pictures/picture} % The first bracket is the picture height, the second is the thickness of the frame around the picture (0pt for no frame)
%\quote{"Faith sees a beautiful blossom in a bulb" - William Arthur Ward}

%----------------------------------------------------------------------------------------

\begin{document}

\makecvtitle % Print the CV title

\renewcommand{\listitemsymbol}{~~~~} % Changes the symbol used for lists

%----------------------------------------------------------------------------------------
%	EDUCATION SECTION
%----------------------------------------------------------------------------------------

\section{Education}

\cvitem{2022-2023}{\textbf{Internship, Complexity Sciences and Evolution Unit:} \href{https://oist.jp/}{Okinawa Institute of Science and Technology}, Okinawa, Japan }
\cvitem{2019-2025}{\textbf{Bachelor of Physics:} \href{https://en.sharif.edu/}{Sharif University of Technology}, Tehran, Iran }
\renewcommand{\listitemsymbol}{~~~~} % Changes the symbol used for lists
\cvlistdoubleitem{CGPA: 3.3/4.0 (4-Point GPA)}{Highest SGPA: 4.0/4.0 (4-Point GPA)}

\cvitem{2012-2019}{\textbf{Secondary School:} \href{https://sampad.gov.ir/}{NODET} Schools (National Organization for Development of Exceptional Talents), Tehran, Iran}

%--------------------------------------
%  Interests
%--------------------------------------
\section{Areas of Interest}

\cvitem{\textbf{}}{Scientific Computation \& Simulations in Complex Systems}

\section{Scientific skills}

\cvitem{\textbf{}}{Adaptive Dynamics, Networks Dynamics, Agent-Based Models, Time-Series Analysis, Machine Learning, Numerical Analysis}

%----------------------------------------------------------------------------------------
%	SKILLS SECTION
%----------------------------------------------------------------------------------------
%\vspace{14mm}
\section{Technical Skills}

\cvitem{Languages}{Julia programming language, Python(NumPy, SciPy, Pandas,...), C/C++/C\#, Matlab}

\cvitem{Softwares}{Bash \& Linux, Git, \LaTeX, Microsoft Office, Blender, Unity (Game engine)}

\cvitem{Others}{HPC, Distributed Computing, and Parallel Algorithms, Volumetric visualization}

%--------------------------------------
%  Projects
%--------------------------------------

\section{Research Experiences}

\cvitem{Jul 2022 - Oct 2022}{\textbf{B.Sc. Research Project (Prof. Rouhani's Group - Physics Department):} \href{http://www.physics.sharif.edu/}{Sharif University of Technology \& Azadi Innovation Factory}, Iran}
\cvitem{Feb 2024 - Ongoing}{\textbf{Extended B.Sc. Research Project (Prof. Rouhani's Group - Physics Department):} \href{http://www.physics.sharif.edu/}{Sharif University of Technology}, Iran}
\cvitem{Project:}{\textbf{Investigating Interdependencies and High-Order interactions of Crypto-Currencies and Fiat-Currencies (Supervisor: \href{https://scholar.google.com/citations?user=vye8xgQAAAAJ&hl=en}{Prof. Shahin Rouhani})}}
\cvitem{Description:}{Collected and analyzed time series data of exchange rates between various crypto and fiat currencies. Processed data using a range of methodologies, including Pearson correlation, causality tests (such as Granger test), and moving averages. Applied stochastic processes, such as Pearson correlation and Markov models, to study exchange rate dynamics and market behavior. Conducted statistical causal modeling, analyzed high-order interactions, and visualized currency valuation networks. Discovered and analyzed communities within high-order interaction graphs, delivering insights into the dependency of fiat currencies on cryptocurrencies, informing policy-making and trading strategies.\newline}
\cvitem{Oct 2022 - Oct 2023}{\textbf{Research Intern (Complexity and Evolution unit, Onsite):} \href{https://www.oist.jp/}{Okinawa Institute of Science and Technology}, Japan}
\cvitem{Oct 2023 - Oct 2024}{\textbf{Research Intern (Complexity and Evolution unit, Remote):} \href{https://www.oist.jp/}{Okinawa Institute of Science and Technology}, Japan}
\cvitem{Oct 2024 - Ongoing}{\textbf{Visiting Researcher (Complexity and Evolution unit, Remote):} \href{https://www.oist.jp/}{Okinawa Institute of Science and Technology}, Japan}
\cvitem{Project:}{\textbf{Prolonged latency as a seasonal adaptation in infectious diseases (Supervisors: \href{https://scholar.google.com/citations?user=rAcGGSgAAAAJ&hl=en}{Prof. Ulf Dieckmann}, \href{https://scholar.google.com/citations?user=PetVfGgAAAAJ&hl=en}{Prof. Ake Brannstrom})}}
\cvitem{Description:}{Developed and applied numerical algorithms in Julia to explore how seasonality influences pathogen latency in infectious diseases. Used Floquet theory and adaptive dynamics for analysis, and conducted simulations and distributed computing experiments. Visualized findings in 2D and 3D, demonstrating how evolutionary pressures favor specific dormancy durations, enabling viral persistence over years. Key contributions include efficient coding, experiment optimization, dataset analysis, and creating publication-ready graphics. This research enhances our understanding of seasonality's role in pathogen behavior.}

%---------------------------------------------------------------------
%	SELECTED COURSES
%------------------------------------------------------------------------

\section{Selected Courses}
\cvitem{Spring 2020}{Introduction to Universe (Dr. B.Mashhoon) - Grade: A}
\cvitem{Spring 2021}{Modeling Statistical Phenomena (Dr. F.Ghanbarnejad) - Grade: A}
\cvitem{Spring 2021}{Thermodynamic and Statistical Physics 1 (Dr. O.Akhavan) - Grade: A}
\cvitem{Fall 2021}{Thermodynamic and Statistical Physics 2 (Dr. O.Akhavan) - Grade: A}
\cvitem{Fall 2021}{Computer Simulation in Physics (Dr. M.R.Ejtehadi) - Grade: A}
\cvitem{Fall 2021}{Computer Simulation in Physics Lab (Dr. M.R.Ejtehadi) - Grade: A}
\cvitem{Fall 2021}{Introduction to Neuroscience (Dr. A. Ghazizadeh) - Grade: B}
\cvitem{Spring 2022}{Complex Systems (Dr. S.Ruhani) - Grade: A}
\cvitem{Spring 2022}{Data Science \& HPC (Dr. H.R.Arian) - Grade: A}
\cvitem{Spring 2024}{Astrophysics Lab (Dr. R.Rezaei) - Grade: A}
\cvitem{Summer 2024}{Electro-Acoustic Lab (Dr. S.Moghimi) - Grade: A}

% --------------------------------------
%  Teaching Experiences
% --------------------------------------

\section{Teaching Experiences}

\cvitem{Spring 2024}{TA of the "Computer Simulation in Physics" course by \href{http://physics.sharif.ir/~phyweb/vaezi/}{Prof. A.Vaezi}}
\cvitem{Links:}{\href{https://github.com/shahmari/Computer-Simulations-in-Physics-2024}{GitHub repository}}

%--------------------------------------
%  Term Projects
%--------------------------------------

\section{Selected Course Projects}

\cvitem{Dec 2020}{\textbf{Simulation of Mechanical Random Walker System:} Python simulation for Analytical Mechanics 1. Calculated particle forces and accelerations, simulating falls and collisions. Derived final location distributions, demonstrating the central limit theorem.}
\renewcommand{\listitemsymbol}{} % Changes the symbol used for lists
\cvlistitem{\textbf{\href{https://GitHub.com/shahmari/Random-Walk-particle-system}{GitHub repository}}}
\cvitem{May 2021}{\textbf{Simulation of Janus Bunch:} Developed a simulation for the Janus Bunch device as part of Analytical Mechanics 2. Numerically solved dynamics of two-phase coupled oscillators using Python, exploring behavior, synchronization, and visualizing system phases and properties.}
\renewcommand{\listitemsymbol}{} % Changes the symbol used for lists
\cvlistitem{\textbf{\href{https://GitHub.com/shahmari/Synchronization-in-janus-bunch}{GitHub repository}}}

\cvitem{May 2021}{\textbf{Investigating the effects of Prevention and Quarantine on SIR:} Term project for Modeling Statistical Phenomena under Dr. F. Ghanbarnejad. Explored prevention and quarantine impacts on SIR models using complex networks and mean field models. Optimized simulation algorithm, transitioning from Python to C for better performance. Successfully executed on HPC cluster, with analysis visualized in Python, revealing effects on disease outbreaks.}
\renewcommand{\listitemsymbol}{} % Changes the symbol used for lists
\cvlistitem{\textbf{\href{https://GitHub.com/shahmari/Prevention-and-Quarantine-on-SIR}{GitHub repository}}}

\cvitem{Jan 2022}{\textbf{Motor Cortex Electrophysiology: Analyzing Macaque Monkey Behavior:} Contributed to a Neuroscience term project by examining electrophysiological data from the motor cortex of macaque monkeys during a reach-to-grasp task. Managed data cleaning, analysis, algorithm development, visualization, and technical aspects, providing insights into brain event sequences.}
\renewcommand{\listitemsymbol}{} % Changes the symbol used for lists
\cvlistitem{\textbf{\href{https://GitHub.com/ali-mahani/neuro-project}{GitHub repository}}}

\cvitem{Mar 2022}{\textbf{Analyzing the Commodity Market Data:} Spontaneous project supervised by Prof. S. Rouhani during the Complex Systems course. Explored modeling commodity prices through time series analysis. Managed data acquisition and cleaning, performed analysis, visualized results, and investigated correlations between market entries, constructing a network to illustrate relationships.}
\renewcommand{\listitemsymbol}{} % Changes the symbol used for lists
\cvlistitem{\textbf{\href{https://GitHub.com/shahmari/Modeling-of-Commodity-Prices}{GitHub repository}}}

\cvitem{Jul 2022}{\textbf{The Effect of Rumor Dynamics on Disease Dynamics:} Collaborated during the Sharif SocioPhysics summer school on a project exploring misinformation's impact on COVID-19 dynamics. Investigated rumor influence on disease spread and mortality using a mean-field model. Provided numerical solutions, adapted code from Julia to Python, and visualized results. Uncovered how misinformation accelerates disease propagation without accurate information.}
\renewcommand{\listitemsymbol}{} % Changes the symbol used for lists
\cvlistitem{\textbf{\href{https://GitHub.com/abbasshojakani/Sharif-SocioPhysics}{GitHub repository}}}
%--------------------------------------
%  SELECTED ASSIGNMENTS
%--------------------------------------

\section{Other Archives}

\cvitem{Spring 2021}{\textbf{Modeling Infectious Diseases} Archive of some of the course assignments}
\renewcommand{\listitemsymbol}{} % Changes the symbol used for lists
\cvlistitem{\textbf{\href{https://GitHub.com/shahmari/Some-minor-effort-in-the-epidemic}{GitHub repository}}}
\cvitem{Fall 2021}{\textbf{Computer Simulation in Physics} Archive of the course assignments}
\renewcommand{\listitemsymbol}{} % Changes the symbol used for lists
\cvlistitem{\textbf{\href{https://GitHub.com/shahmari/ComputationalPhysics-Fall2021}{GitHub repository}}}
\cvitem{Fall 2023}{\textbf{Network Science} Archive of some of the course assignments and course project}
\renewcommand{\listitemsymbol}{} % Changes the symbol used for lists
\cvlistitem{\textbf{\href{https://github.com/shahmari/Network-Science-course}{GitHub repository}}}
\cvitem{Spring 2024}{\textbf{Astrophysics Laboratory} Archive of the course assignments}
\renewcommand{\listitemsymbol}{} % Changes the symbol used for lists
\cvlistitem{\textbf{\href{https://github.com/shahmari/Astrophysics-Laboratory}{GitHub repository}}}
\cvitem{Summer 2024}{\textbf{Electro-Acoustic Laboratory} Archive of the course assignments}
\renewcommand{\listitemsymbol}{} % Changes the symbol used for lists
\cvlistitem{\textbf{\href{https://github.com/shahmari/Electro-Acoustic-Lab}{GitHub repository}}}
% %----------------------------
%   Extracurricular Ativities
%----------------------------

\section{Community}
\cvitem{2020-2021}{Main member of the Committee, Public relations manager, and Graphist of Sharif Physics Student Scientific Association, \href{https://spssa.ir/}{Link}}
\cvitem{2020-2021}{Head of Scientific News-Reading of Department, \href{https://t.me/khabarkhaani}{Link}}
\cvitem{2020-2021}{Co-founder and lecturer of "Lambda Scientific Circle", \href{https://t.me/lambda_circle}{Link}}
\cvitem{Fall 2020}{Organizer of the "Introducing the Branches of Physics" Conference, \href{https://bit.ly/3O7duaF}{Link}}
\cvitem{Spring 2021}{Organizer of the "Introduction to Complex Systems" event, \href{https://t.me/anjoman_elmi_phys_sut/2062}{Link}}
\cvitem{Summer 2021}{Member of the executive team at Sharif Socio-Physics School, \href{http://physics.sharif.edu/~ssp2021/}{Link}}
\cvitem{Winter 2022}{Lecturer of the SPSSA Julia Workshop, \href{https://bit.ly/3ruO1ht}{Link}}

\pagebreak

%--------------------------------------
%  References
%--------------------------------------

% \section{References}

% \cvitem{1.}{Dr. Ulf Dieckmann}
% \cvitem{}{\textbf{Full Professor and Director, OIST(Japan) and IIASA(Austria)}}
% \cvitem{}{Email: \href{mailto:ulf.dieckmann@oist.jp}{ulf.dieckmann@oist.jp}, Links: \href{https://groups.oist.jp/cse/ulf-dieckmann}{Web Page}, \href{https://scholar.google.com/citations?user=rAcGGSgAAAAJ&hl=en}{Google Scholar}}
% \cvitem{}{Relationship: \textbf{Research Supervisor}}
% \cvitem{2.}{Dr. Åke Brännström}
% \cvitem{}{\textbf{Full Professor of Mathematics, Umeå University(Sweden)}}
% \cvitem{}{Email: \href{mailto:ake.brannstrom@umu.se}{ake.brannstrom@umu.se}, Links: \href{https://www.umu.se/en/staff/ake-brannstrom/}{Web Page}, \href{https://scholar.google.com/citations?user=PetVfGgAAAAJ&hl=en}{Google Scholar}}
% \cvitem{}{Relationship: \textbf{Research Supervisor}}
% \cvitem{3.}{Dr. Shahin Rouhani}
% \cvitem{}{\textbf{Full Professor of Physics, Sharif University of Technology and IPM}}
% \cvitem{}{Email: \href{mailto:rouhani@ipm.ir}{rouhani@ipm.ir}, Links: \href{http://physics.sharif.ir/~phyweb/shahin-rouhani/}{Web Page}, \href{https://scholar.google.com/citations?user=vye8xgQAAAAJ&hl=en}{Google Scholar}}
% \cvitem{}{Relationship: \textbf{Professor and Research Supervisor}}
% \cvitem{4.}{Dr. Mohammad Reza Ejtehadi}
% \cvitem{}{\textbf{Full Professor of Physics, Sharif University of Technology and IPM}}
% \cvitem{}{Email: \href{mailto:ejtehadi@sharif.edu}{ejtehadi@sharif.edu}, Links: \href{http://softmatter.physics.sharif.edu/people/group-leader/prof-reza-ejtehadi/}{Web Page}, \href{https://scholar.google.com/citations?user=FKH-RL4AAAAJ&hl=en}{Google Scholar}}
% \cvitem{}{Relationship: \textbf{Professor and Research Supervisor}}
% \cvitem{5.}{Dr. Omid Akhavan}
% \cvitem{}{\textbf{Associate professor of Physics, Sharif University of Technology}}
% \cvitem{}{Email: \href{mailto:oakhavan@sharif.edu}{oakhavan@sharif.edu}, Links: \href{http://physics.sharif.ir/~phyweb/omid-akhavan/}{Web Page}, \href{https://scholar.google.com/citations?user=Jkmd00gAAAAJ&hl=en}{Google Scholar}}
% \cvitem{}{Relationship: \textbf{Professor}}
% \cvitem{6.}{Dr. Saman Moghimi-Araghi}
% \cvitem{}{\textbf{Associate professor of Physics, Sharif University of Technology}}
% \cvitem{}{Email: \href{mailto:samanimi@sharif.edu}{samanimi@sharif.edu}, Links: \href{http://physics.sharif.ir/~phyweb/saman-moghimi-araghi/}{Web Page}, \href{https://www.researchgate.net/profile/Saman-Moghimi-Araghi}{ResearchGate}}
% \cvitem{}{Relationship: \textbf{Professor}}

% \begin{center}
%     \textbf{!! Under Construction !!}

%     .

%     .

%     .
% \end{center}

\end{document}